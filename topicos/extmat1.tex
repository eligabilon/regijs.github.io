\documentclass[a4paper,12pt]{article}
\usepackage[brazil]{babel}
\usepackage{graphicx,color}
\usepackage[T1]{fontenc}
\usepackage{amsthmc,amsfonts}
\usepackage{hvdashln}
%\usepackage{fancybox}
\setlength{\textwidth}{6 in}
\setlength{\textheight}{23 cm}
\evensidemargin 0 in
\oddsidemargin 0.25 in
\hdashlinewidth=2pt
\hdashlinegap=2pt
\sloppy

\renewcommand{\rmdefault}{cmss}
\bibliographystyle{plain}
\pagestyle{empty}
\begin{document}
%\nocite{kolman,anton,strang2,morgado}
\title{T�picos em Matem�tica\\
        {\small Carga Hor�ria: 90 horas}}
\author{Proposta do Prof. Reginaldo J. Santos\\
         DMat-ICEx-UFMG}
\maketitle


\section*{Ementa}
Introdu��o ao MATLAB, Introdu��o ao programa de editora��o
eletr�nica de textos matem�ticos \LaTeX, Matem�tica
Financeira e Aplica��es de �lgebra Linear com o uso do
MATLAB.

\section*{Programa}
\begin{enumerate}
\item Introdu��o ao MATLAB.
Vamos mostrar como este programa pode fazer o computador
trabalhar como uma m�quina de calcular cient�fica
sofisticada.
\item Introdu��o ao \LaTeX.
Este � o programa usado pelos matem�ticos, f�sicos e outros cientistas
para escrever os artigos cient�ficos. � o programa adequado
para editora��o eletr�nica de textos que contenham f�rmulas.
Ele pode ser usado tamb�m para escrever home pages.
\item Matem�tica Financeira.
Vamos mostrar entre outras coisas como encontrar os valores
das presta��es em financiamentos que est�o sendo pr�ticados
hoje no mercado, como a tabela price, o sistema SAC e
sistema SACRE, usado pelo sistema financeiro da habita��o.
Al�m disso, vamos mostrar como calcular a taxa de juros que
est� realmente sendo praticada nos financiamentos.
\item Aplica��es de �lgebra Linear.
%\begin{enumerate}
Vamos estudar como podemos  construir curvas e superf�cies
que passam por pontos dados. Ou como no caso em que os
pontos s�o obtidos experimentalmente, como usar Quadrados
M�nimos e Splines para encontrar curvas e superf�cies suaves
que melhor representam os pontos.

%\end{enumerate}
\end{enumerate}

%\bibliography{bibli}


\end{document}
