% Este � um pequeno arquivo fonte para o LaTeX (vers�o de 12 de mar�o de 1998)
%
% Use este arquivo como modelo para fazer seus pr�prios arquivos  LaTeX.
% Tudo que est� � direita de um  %  � um coment�rio e � ignorado pelo LaTeX.

% CUIDADO!  Os 10 caracteres abaixo t�m significado especial:
%                &   $   #   %   _   {   }   ^   ~   \

\documentclass[a4paper,12pt]{article}% Seu arquivo fonte precisa conter
\usepackage[brazil]{babel}           % estas quatro linhas
\usepackage[T1]{fontenc}             % al�m do comando \end{document}
\begin{document}                     % no fim.


\section{Texto e Comandos} % Este comando faz o t�tulo da se��o.

Palavras s�o separadas por um ou mais espa�os. Paragrafos
s�o separados por uma ou mais linhas em branco. A sa�da n�o
� afetada por espa�os extras ou por linhas em branco extras.

%Aspas s�o digitadas assim:
``Texto entre aspas''.

%Texto enfatizado deve ser digitado como:
\textit{Isto est� em it�lico}.

%Texto em negrito deve ser digitado como:
\textbf{Isto est� em negrito}.

\subsection{Um aviso}  % Este comando faz o t�tulo da subse��o.

Lembre-se, n�o escreva os 10 caracteres especiais
%                &   $   #   %   _   {   }   ^   ~   \
\& \$ \# \% \_ \{ \} \textasciicircum \ \textasciitilde
\ \textbackslash \ exceto como um comando! Eles s�o impressos
com os comandos
%Este comando inicia um ambiente ``verbatim''(ao p� da letra)
\begin{verbatim}
\& \$ \# \% \_ \{ \} \textasciicircum \textasciitilde \textbackslash.
\end{verbatim}   %Este comando termina o ambiente ``verbatim''.
                 %Tudo que estiver dentro do ambiente � impresso
                 %exatamente como � digitado
A maioria dos comandos do \LaTeX \ s�o iniciados com o
caracter \textbackslash. Uma \textbackslash \ sozinha produz
um espa�o.
\end{document}   % O arquivo fonte termina com este comando.
